

\documentclass[11pt,a4paper,skipsamekey]{moderncv}      
\usepackage[english]{babel}



\moderncvstyle{classic}                            
\moderncvcolor{green}                            

% character encoding
\usepackage[utf8]{inputenc}                     
\usepackage{comment}

% adjust the page margins
\usepackage[scale=0.75]{geometry}

% personal data
\name{Nathan B.}{Corral}
\address{Frankenweg 26A 53225, Bonn}
\phone[mobile]{+49 160 9178 1918}               
\email{nathan.b.corral@gmail.com}                             

\begin{document}
	\recipient{To:}{Teckentrup GmbH \& Co. KG \\Industriestraße 50 \\33415 Verl}
	\date{\today}
	\opening{Sehr geehrte Frau Zier,}
	\closing{Mit freundlichen Grüßen,}
	%\enclosure[Attachment]{CV}
	\makelettertitle
	
	mit großem Interesse habe ich Ihre Ausschreibung für die Position als Junior KI-Entwickler gelesen. Als Informatiker mit einem Masterabschluss und fundierten Kenntnissen in den Bereichen Künstliche Intelligenz und Maschinelles Lernen bin ich überzeugt, einen wertvollen Beitrag zu Ihren innovativen Digitalisierungsprojekten leisten zu können.
	
	In meiner Masterarbeit habe ich eine Technik zur Videogenerierung entwickelt, die Transformer-Netzwerke verwendet, um das nächste Bild eines Videos durch Abtastung aus einer erlernten Verteilung möglicher Zukünfte vorherzusagen. Diese Methode habe ich auf die Generierung von menschlichen Bewegungen angewandt, um die Aktion eines 3D-Skeletts fortzusetzen. Die Arbeit, die mit Bestnoten bewertet wurde, gab mir praktische Erfahrung in der Entwicklung und Optimierung moderner Deep-Learning-Architekturen und deren Anwendung auf komplexe KI-Probleme. Diese Fähigkeiten, kombiniert mit meiner Begeisterung für innovative Technologien, machen mich zu einem starken Kandidaten für Ihr Team.
	
	Ich freue mich darauf, Sie und Ihr Team bei der Weiterentwicklung Ihrer KI-Projekte zu unterstützen. Über die Möglichkeit zu einem persönlichen Gespräch, in dem ich mehr über die Position und Ihre Erwartungen erfahren darf, würde ich mich sehr freuen.
	
	
	\vspace{0.5cm}
	\makeletterclosing
	
	
	
	
\end{document}