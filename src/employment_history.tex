\begin{rubric}{Berufserfahrung}

%
%  Student Assistant Postion
%
\entry*[] \textbf{Humanoid Robots Lab -- Universität Bonn} \hfill \textit{Wissenschaftlicher Mitarbeiter} \newline  
09.2021 -- 09.2022 \hfill Bonn, Deutschland \newline  
\vspace{\CVItemizeHeaderSpacing} \begin{itemize}
	\setlength{\itemsep}{\CVItemizeSpacing}  
	\item Mitgewirkt an Forschung und Veröffentlichungen in Bereich "Personalized Robot Navigation".  
	\item Programmierung der ROS-Schnittstelle für die 3D-Lokalisierung von Menschen mit einer RGBD-Kamera unter Nutzung von Deep Learning und Implementierung dieser Funktion auf einem realen Roboter zur autonomen Navigation.  
	\item Verwendung des fotorealistischen Simulators iGibson (PyBullet-Backend) zur Generierung von Daten für einen Deep-Reinforcement-Learning-basierten Path Planning Algorithm.  
	\item Aufbau und Durchführung einer Nutzerstudie zur Bewertung der Mensch-Roboter-Interaktion mit einem VR-Headset und anschließender Umsetzung auf realer Roboterhardware.  
\end{itemize}

%
% Head Rush Technologies
%
\entry*[] \textbf{Head Rush Technologies} \hfill \textit{Vertragsingenieur} \newline  
12.2019 -- 04.2020 \hfill Boulder, USA \newline  
\vspace{\CVItemizeHeaderSpacing} \begin{itemize} 
	\setlength{\itemsep}{\CVItemizeSpacing}  
	\item Vertragsarbeit zur Entwicklung der Firmware auf einem ATmega328PB-Mikrochip für ein Proof-of-Concept-System.  
	\item Arbeit umfasste die Programmierung eines durch Interrupts ausgelösten Zahnrad-Sensors, RS485-Kommunikation, einer PWM-gesteuerten Bremse sowie Logik für endliche Zustandsautomaten.  
	\item Durchführung von Feldtests und Erstellung der Projektdokumentation.  
	\item Der Erfolg dieses Prototyps führte zu einer weiteren Entwicklung, die letztendlich als ihre „Catch-and-Hold-Technology“ veröffentlicht wurde.  
\end{itemize}


%
% Aqronos
%
\entry*[] \textbf{Aqronos} \hfill \textit{Softwareentwickler} \newline  
11.2018 -- 12.2019 \hfill Denver, USA \newline  
\vspace{\CVItemizeHeaderSpacing} \begin{itemize} 
	\setlength{\itemsep}{\CVItemizeSpacing}  
	\item Entwicklung von ROS-Nodes zur Visualisierung des LiDAR-Prototyps des Unternehmens.  
	\item Strukturierung von UDP-Paketen und Programmierung beider Seiten der Sende- und Empfangsmodule.  
	\item Interaktion mit einer REST-API auf dem eingebetteten System zur Konfiguration von Hyperparametern.  
	\item Filterung von Punktwolken und Gruppierung von Objekten mit der C++ Point Cloud Library.  
\end{itemize}



%
% CE
%
\entry*[] \textbf{Creative Edge LLC} \hfill \textit{Softwareentwickler} \newline  
08.2017 -- 09.2018 \hfill Denver, USA \newline  
\vspace{\CVItemizeHeaderSpacing} \begin{itemize}  
	\setlength{\itemsep}{\CVItemizeSpacing}  
	\item Entwicklung von Anwendungen für das Kryptowährungs-Mining unter Windows und Linux.  
	\item Erstellung von Software zur Verwaltung von Betriebssystemtreibern, Systemkonfigurationen und Tools von Drittanbietern.  
\end{itemize}

\end{rubric}
