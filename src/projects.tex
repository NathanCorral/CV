\begin{rubric}{Projects}

\entry*[] \textbf{ROS 2 Whisper} \hfill 2024 \newline
\emph{Maintainer} \hfill  \href{https://github.com/ros-ai/ros2_whisper/blob/main/doc/harry_potter_sample.gif}{Video}, \href{https://github.com/ros-ai/ros2_whisper}{\faGithub Source} \newline
\vspace{\CVItemizeHeaderSpacing} \begin{itemize}[leftmargin=*, rightmargin=1cm]
	\setlength{\itemsep}{\CVItemizeSpacing}
	\item Extended this open source project to support boarder-less, live transcription.
	\item Implemented the C++ code to place special attention on code efficiency and scalability.
\end{itemize}

\entry*[] \textbf{ROS 2 Computer Vision} \hfill 2024 \newline
\emph{Author} \hfill \href{https://github.com/NathanCorral/ROS-HF-Vision/blob/main/doc/gifs/ex_german_roads.gif}{Video}, \href{https://github.com/NathanCorral/ROS-HF-Vision/tree/main}{\faGithub Source} \newline
\vspace{\CVItemizeHeaderSpacing} \begin{itemize}[leftmargin=*, rightmargin=1cm]
	\setlength{\itemsep}{\CVItemizeSpacing}
	\item Designed a ROS 2 pipeline to run multiple Computer Vision (CV) tasks (Object Detection, Per-Pixel Segmentation) in parallel.  
	\item Automatically download modern CV models (such as DETR, Maskformer).  
	\item Re-index the model output labels, which may be trained on different datasets, into a universal database.  
	\item Run the pipeline on both live camera feed and a dataset, which allowed time comparisons between the asynchronous running of multiple models.
\end{itemize}
\begin{comment}
\entry*[] \textbf{Semantic Search using Facebook AI Similarity (FAISS)} \hfill 2024 \newline \emph{Author} \hfill \href{https://github.com/NathanCorral/Hugging-Face-FAISS-Semantic-Search}{\faGithub Source} \newline
\vspace{\CVItemizeHeaderSpacing} \begin{itemize}[leftmargin=*, rightmargin=1cm]
	\setlength{\itemsep}{\CVItemizeSpacing}
	\item Implemented the first steps in Retrieval-Augmented Generation (ending before "Generation").
	\item Programmed web-scraping, dataset embedding, and similarity comparisons to recover matches in the dataset from a natural language query.
\end{itemize}

\begin{comment}
\entry*[2021] \textbf{Temporal Convolutional Network} \newline
As part of a class project, we re-implemented the Multi-Stage Temporal Convolutional Network\footnote{Y. Abu Farha et al., "MS-TCN: Multi-Stage Temporal Convolutional Network for Action Segmentation." CVPR 2019.} in PyTorch.  This project achieved: \newline
\vspace{\CVItemizeHeaderSpacing} \begin{itemize}
	\setlength{\itemsep}{\CVItemizeSpacing}
	\item Understanding state-of-the-art (2019) computer vision networks on video action classification.
	\item Training and testing the model on a subset of an actively used dataset ($\approx$30\% of the Breakfast Actions Dataset).
	\item Verifying results reported in the paper (66\% accuracy).
\end{itemize}
\end{comment}
\begin{comment}
\entry*[2016] \textbf{Tower of Hanoi} \newline
I programmed a robot arm to play the Tower of Hanoi game with full automation.  The project goals consisted of:  \newline
\vspace{\CVItemizeHeaderSpacing} \begin{itemize}
	\setlength{\itemsep}{\CVItemizeSpacing}
	\item Calculating the Inverse Kinematics for a ROS Controlled Rhino RX2 arm, and programming them in C++.
	\item Attaching a down-facing camera and building a computer vision node to identify blocks in the robot plane.
	\item Establishing tower centers and programming the Tower of Hanoi game logic.
\end{itemize}
\end{comment}
\end{rubric}

