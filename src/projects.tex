\begin{rubric}{Projects}

\entry*[2024] \textbf{ROS 2 Whisper} \hfill \href{https://github.com/NathanCorral/ros2_whisper/blob/dev-code-cleanup/doc/harry_potter_sample.gif}{Video}, \href{https://github.com/NathanCorral/ros2_whisper/tree/dev-code-cleanup}{\faGithub Source} \newline
As an extension of this open source project, I implemented boarder-less, live audio transcription. Written in C++, my code contribution emphasizes: \newline
\vspace{\CVItemizeHeaderSpacing} \begin{itemize}
	\setlength{\itemsep}{\CVItemizeSpacing}
	\item Scalability, using both inheritance and composition in object-oriented programming behavior.
	\item Efficiency, through intentional memory management, thread-safe callbacks and work splitting across multiple nodes.
	\item Simplicity, in the well thought-out implementation of complex merging algorithms.
\end{itemize}

\entry*[2024] \textbf{ROS 2 Computer Vision} \hfill \href{https://github.com/NathanCorral/ROS-HF-Vision/blob/main/doc/gifs/ex_german_roads.gif}{Video}, \href{https://github.com/NathanCorral/ROS-HF-Vision/tree/main}{\faGithub Source} \newline
Running multiple computer vision models (DETR, Maskformer) trained across different datasets/tasks on a live camera feed introduces several implementation challenges. This Python repository presents a solution for: \newline
\vspace{\CVItemizeHeaderSpacing} \begin{itemize}
	\setlength{\itemsep}{\CVItemizeSpacing}
	\item Downloading and running state-of-the-art models from Hugging Face as asynchronous ROS 2 nodes.
	\item Hosting a label server for re-addressing model outputs into a global database.
	\item Displaying segmentation masks and bounding boxes as a Matplotlib animation.
	\item Publishing dataset images for repeatable evaluation of CV models.
\end{itemize}

\entry*[2024] \textbf{Semantic Search using Facebook AI Similarity (FAISS)} \hfill \href{https://github.com/NathanCorral/Hugging-Face-FAISS-Semantic-Search}{\faGithub Source} \newline
\qquad This project implements the first steps in Retrieval-Augmented Generation (RAG) (stopping at "Generation").  I perform web scraping, dataset/query embedding, and similarity scoring to lookup data from a natural language query.  
%\vspace{\CVItemizeHeaderSpacing} \begin{itemize}
%	\setlength{\itemsep}{\CVItemizeSpacing}
%	\item Web-scraping the Hugging-Face Github to create a dataset of issues.
%	\item Using a pre-trained Large Language Model to create embeddings over text in the issues.
%	\item Computing the embedding of a natural language query.
%	\item Retrieval of similar issues by computing the Faiss.
%\end{itemize}


\begin{comment}
\entry*[2021] \textbf{Temporal Convolutional Network} \newline
As part of a class project, we re-implemented the Multi-Stage Temporal Convolutional Network\footnote{Y. Abu Farha et al., "MS-TCN: Multi-Stage Temporal Convolutional Network for Action Segmentation." CVPR 2019.} in PyTorch.  This project achieved: \newline
\vspace{\CVItemizeHeaderSpacing} \begin{itemize}
	\setlength{\itemsep}{\CVItemizeSpacing}
	\item Understanding state-of-the-art (2019) computer vision networks on video action classification.
	\item Training and testing the model on a subset of an actively used dataset ($\approx$30\% of the Breakfast Actions Dataset).
	\item Verifying results reported in the paper (66\% accuracy).
\end{itemize}


\entry*[2016] \textbf{Tower of Hanoi} \newline
I programmed a robot arm to play the Tower of Hanoi game with full automation.  The project goals consisted of:  \newline
\vspace{\CVItemizeHeaderSpacing} \begin{itemize}
	\setlength{\itemsep}{\CVItemizeSpacing}
	\item Calculating the Inverse Kinematics for a ROS Controlled Rhino RX2 arm, and programming them in C++.
	\item Attaching a down-facing camera and building a computer vision node to identify blocks in the robot plane.
	\item Establishing tower centers and programming the Tower of Hanoi game logic.
\end{itemize}

\end{comment}

\end{rubric}

